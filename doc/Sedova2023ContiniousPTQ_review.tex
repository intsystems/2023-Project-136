\documentclass[DIV=16]{scrartcl}

\usepackage[english, russian]{babel}

\pagestyle{empty}

% fonts
\renewcommand{\rmdefault}{cmss}
\renewcommand{\ttdefault}{cmss}

\pagestyle{empty}

\begin{document}
	
	\begin{LARGE}
	\begin{center}
		Рецензия на рукопись \\
		Flexible continuous modification for SOTA \\
		post training quantization methods \\
		Автор Седова Анна
	\end{center}
	\end{LARGE}

	\textit{\textbf{Список неточностей, опечаток:}}

	\textbf{Аннотация}
	
	Строка 1 - возможно, стоит заменить "resource constrained" $\rightarrow$ "resource-constrained".
	
	Строка 5 - "accuracy of model" $\rightarrow$ "accuracy of the model"
	
	\textbf{1. Вступление}
	
	Строка 2 - после (PTQ), видимо, опечатка - стоит слово "pos".
	
	\textbf{2. Постановка задачи}
	
	Стоит пронумеровать формулы
	
	Стоит пояснить обозначение скобок в выражении $\lfloor\frac{v}{\alpha}\rceil$ в первой формуле
	
	\textbf{3. Quantization modification}
	
	3 абзац, 2 строчка - не нужна запятая в "more, than"
	
	3.4, 1 абзац, 1 строка - видимо, пропущено слово "highest": "we denote the weight as $w_p$, and the lowest as $w_n$"
	
	3.4 Немного неровно отображается формула 1
	
	\textbf{4. Experiments}
	
	1 абзац, строка 1. "conduct two experiments to conduct the effectiveness" $\rightarrow$ "сonduct two experiments to estimate the effectiveness" 
	
	4.2 В таблице не отображаются значения в ячейках \\
	
	\textbf{\textit{Доработать:}}
	
	Исправить опечатки, дописать заключение \\
	
	\textbf{\textit{Комментарий о работе в целом}}
	
	Работу легко и приятно читать, присутствует четкая постановка задачи, цель работы. Методы и их преимущества описаны понятным языком. Возможно, для большей целостности работы стоит добавить больше ссылок между частями статьи, например, в секции со сравнением качества методов - на их описания \\
	
	Рецензент:
	
	Швейкин Денис
	
	
\end{document}
